%\documentclass[12pt]{report}
%
%% some macros
%% Here, you can define your own macros. Some examples are given below.

\newcommand{\R}[0]{\mathds{R}} % real numbers
\newcommand{\Z}[0]{\mathds{Z}} % integers
\newcommand{\N}[0]{\mathds{N}} % natural numbers
\newcommand{\C}[0]{\mathds{C}} % complex numbers
\newcommand{\bm}[1]{{\boldsymbol{{#1}}}} % vector
\newcommand{\mat}[1]{{\boldsymbol{{#1}}}} % matrix

\newcommand{\E}[1]{\mathbb{E}_{#1}} % Expectation
\newcommand{\Pd}[1]{\mathbb{P}_{#1}} % Probability Distribution

%%%%%%%%%%%%%%%%%%%%%%%%%%%%%%%%%%%%%%%%%%
% University Assignment Title Page 
% LaTeX Template
% Version 1.0 (27/12/12)
%
% This template has been downloaded from:
%%%%%%%%%%%%%%%%%%%%%%%%%%%%%%%%%%%%%%%%%
%----------------------------------------------------------------------------------------
%	PACKAGES AND OTHER DOCUMENT CONFIGURATIONS
%----------------------------------------------------------------------------------------

\usepackage[nottoc,notlot,notlof]{tocbibind}
\usepackage[T1]{fontenc}
%\usepackage{hyperref}
\usepackage{indentfirst}
\usepackage{amsmath}
\usepackage{amssymb}
\usepackage{graphicx}
\usepackage{subcaption}
\usepackage{geometry}
 \geometry
 {
     a4paper,
     left=25mm,
     right=25mm,
     top=30mm,
     bottom=30mm,
 }

% \usepackage[a4paper,hmargin=2.8cm,vmargin=2.0cm,includeheadfoot]{geometry}
\usepackage{textpos}
\usepackage{natbib} % for bibliography
%\usepackage{tabularx,longtable,multirow,subfigure,caption}
\usepackage{fncylab} %formatting of labels
\usepackage{fancyhdr} % page layout
\usepackage{url} % URLs
\usepackage[english]{babel}
%\usepackage{dsfont}
\usepackage{epstopdf}
\usepackage{backref} % needed for citations
\usepackage{array}
\usepackage{latexsym}
\usepackage[pdftex,pagebackref,hypertexnames=false,colorlinks]{hyperref} % provide links in pdf
 
\hypersetup{pdftitle={},
  pdfsubject={}, 
  pdfauthor={},
  pdfkeywords={}, 
  pdfstartview=FitH,
  pdfpagemode={UseOutlines},% None, FullScreen, UseOutlines
  bookmarksnumbered=true, bookmarksopen=true, colorlinks,
    citecolor=black,%
    filecolor=black,%
    linkcolor=black,%
    urlcolor=black}
 
\usepackage[all]{hypcap}
\frenchspacing

\DeclareMathOperator{\tr}{tr}
%
%\begin{document}

\externaldocument{lit_review}

\chapter{Blendshape Classification}

This chapter shall discuss the problem of word level classification from blendshape parameters obtained from a dataset created with the VOCA model \cite{Cudeiro2019} processing audio from the LRW dataset \cite{Chung2016}.
Classification is performed with a Convolutional Neural Network architecture on 500 labels.
The model architecture, how performance is assessed and how the model has been tuned to maximise performance on a held out validation set is also discussed. 

\section{Problem Definition}
As with traditional lip reading problems, the end goal of VSR models using 3D temporal data is to correctly classify variable length input sequences at a sentence level.
However, before this problem can be tackled, the simpler problem of word level classification should be first approached to ensure that this is possible, before moving on to more advanced problems.
If word level classification cannot be achieved, it may be concluded that there is not a sufficient amount of information encoded in 3D temporal models for more complex problems to be solved with the current set of assumptions.

These assumptions include:
\begin{enumerate}
    \item Provided an appropriate dataset of 3D temporal facial scans of subjects speaking, a set of blendshape parameters which realistically represent the motions of speech can be found. \label{assumption:class_1}
    \item The blendshape parameters represent an encoding from which a classification model is able to make word level predictions from. \label{assumption:class_2}
\end{enumerate}

\section{Generation of a 3D Lip Reading Dataset}
In order to train any model, an appropriate dataset must exist upon which the model can be trained.
An appropriate dataset for word level prediction from blendshape parameters would consist of:
\begin{itemize}
   \item A large enough vocabulary that single instances of words with unique phonemes do not exist as this would allow classification of these words too simple.
   \item Facial mesh recordings for all samples in the dataset, all of with brought into alignment to remove movement not related to speech.
   \item Principal blendshape axis which can be shown to contain useful interpolation for speech.
   \item The blendshape parameters corresponding to each sample for the given blendshape axis.
\end{itemize}

As discussed in section \ref{3D Datasets}, currently there exists no public datasets which have been created for the purpose of lip reading. 
In addition to this, the datasets which contain 3D temporal data of speaking facial scans are either not public in the case of Karras's model \cite{Karras2017a} or are unlabelled and too small to train a model effectively as is the case with the VOCASET \cite{Cudeiro2019}.
Both of these datasets also contain a large vocabulary, with too few samples of words.
The VOCASET dataset could be labelled with the use of an automatic speech recognition model such as DeepSpeech as suggested in section \ref{3D Datasets}, however as the dataset was constructed to contain as much variation in speech sounds, so a degree of word repetition would offer little benefit.

In order to tackle the issue of an appropriate dataset not existing, two solutions exist.
The ideal solution would be to capture subjects speaking words from a corpus using a 3D capture equipment, this solution was not a viable option for this project due to the specialised equipment required and the data processing required after data capture had been performed.
Another means of constructing the dataset would be to generate the data with an existing model which can convert an appropriate audio dataset to a series of facial meshes from which blendshape axis can be constructed and parameters recovered.


\section{Data Preprocessing}

\section{Model Architecture}

\section{Results}

\section{Discussion}

%\bibliographystyle{unsrt}
%\bibliography{ref}
%\end{document}